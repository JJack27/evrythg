\documentclass[12pt,a4paper]{article}
\usepackage{amsmath}  % required for {operatorname}
\usepackage{amsfonts} % required for {mathbb}
% \usepackage{trfsigns} % required for laplace transformation signs, not available in EICOM
% \usepackage{hyperref} % alows \href{url}{Label}, not yet made available on arch
\usepackage{siunitx} 
\usepackage{pythontex}
\begin{document}

\tableofcontents

% set up sympy
\begin{sympycode}
sys.path.append(r'C:\Users\Murad\evrythg')
from sympy.physics.units import *
# clean_sym.py
from clean_sym import *
# _sympy.py 
from _sympy import _idx
# clean_numeric_values.py
import clean_numeric_values as n
\end{sympycode}

% set up pylab
\begin{pylabcode}
sys.path.append(r'C:\Users\Murad\evrythg')
from numpy import array
from sympy.geometry import Point
from sympy.geometry import Triangle
# wrpr.py
from wrpr import _line
from wrpr import _scatter
from wrpr import _arrow
from wrpr import _pp
from cg import  auw
# clean_sym.py
import clean_sym as sym
# clean_numeric_values.py
import clean_numeric_values as n
from sap import *
\end{pylabcode}


\section{Task description}
Model the dynamics of a rigid body in an environment partly filled with water.
\section{Model description}
Assume that a rigid body is specified by its' geometry $\s{G}$ and density $\s{rho}$, both assumed not to change over time.    In the following positions and directions are described using a right handed coordinate system aligned such, that the $\s{z}$-Axis is collinear to line of action of gravity.  Unit vectors along the axes are given as
\begin{equation*}
\s{[i, j, k]} = \s{eye(3)}
\end{equation*}
and general positions are given by coordinates vectors $\s{r}$.  

When infinitesimal volumes $\s{Vi}$ in the body are asssumed to have constant densities $\s{rhoi}$ infinesimal masses $\s{Mi}$ are subject to weight forces $\s{Wi}$ as
\begin{equation*}
\s{Eq(Mi, rhoi * Vi)} \;;\; \s{Wi} =  \s{g * k} \s{Mi} \;;\; \s{Eq(g, 9.81 * meter/second**2)}
\end{equation*}
\begin{equation}
\s{M} = \int\limits_{\s{G}} \s{Mi} \;;\; \s{W} = \int\limits_{\s{G}} \s{Wi}
\label{eq:maw}
\end{equation}
If the body displaces a volume $\s{D}$ of water it experiences an additional force
$$\s{B} = \s{g * rho_W *  k}\int\limits_{\s{D}} \s{Di} \;;\; \s{Eq(rho_W, 1000 * kilo * gram/meter**3)}$$
The points of application $\s{(C_B, C_W)}$ for the respective forces $\s{(B, W)}$ are
\begin{equation}
	\s{C_W} = \int\limits_{\s{G}} \s{Mi * ri / M} \;;\; \s{C_B} = \int\limits_{\s{B}} \s{ri} 
\label{eq:poa}
\end{equation}
The body would be floating in equilibrium if
$$\s{Eq(W, B)} \;;\; \s{Eq(C_W, C_B)}$$
and no other forces were are present.  
However in general we need to assume that it is affected by a total force
\begin{equation*}
\s{Eq(F, X + W + B)}
\end{equation*}
where $\s{X}$ accounts for any unknown forces.
To describe system dynamics we introduce a state vector $\s{s}$.
In general
\begin{equation}
	\begin{split}
		\s{Eq(F, f)} \\
		\s{Eq(sddot, h)} \\
\end{split}
	\label{eq:eom}
\end{equation}

\section{Implementation}
To solve equations (\ref{eq:maw}, \ref{eq:poa}) we require indicator functions $\mathds{1}_{(.)}(.)$, to decide if any coordinate $\s{r}$ is occupied by a infinitesimal volume $\s{Vi}$ at time $\s{t}$ and if so which density $\s{rhoi}$ is to be assigned to $\s{Vi}$.
As such functions are hard to define for the general case we start with simplified setups.  
If the rigid body is described by a planar triangle and $\s{X}$ is neglected we obtain the forces illustrated in Figure \ref{fg:fbda}
\begin{pylabcode}
o = Point(-1, -1)
p = Point(1, -1)
q = Point(0, 1)
T = Triangle(o, p, q).rotate(pi/5.)
C_W = array([T.centroid.x, T.centroid.y]).astype(float)
D = auw(T)
C_B = array([D.centroid.x, D.centroid.y]).astype(float)
g = array([0, -1])
W =  float(T.area) * n.rho 
B =  - float(D.area) * 1 
figure()
_pp(T)
_pp(D, color='g')
_arrow(C_W, g*W, label='$%s$' %sym.W)
_arrow(C_B, g*B, label='$%s$' %sym.B, color='g')
_scatter(C_W, color='b', label='$%s$' %sym.C_W)
_scatter(C_B, color='g', label='$%s$' %sym.C_B)
legend(scatterpoints=1)
grid()
savefig('triangle.pdf')
\end{pylabcode}
\begin{pylabcode}
close('all')
T = R_2D(pi/5, array([[-1, -1], [1, -1], [0, 1]]))
p_i = uspit(n.N, T)
v_i =  A_T(T) / n.N * ones((p_i.shape[0], 1))
w_i, v_w = dw(p_i, v_i)  
_line(T)
C_B = mean(w_i, axis=0)
C_W = mean(p_i, axis= 0)
g = array([0, -1])
B = sum(v_w) * -1. * g
W = sum(v_i) * n.rho * g
_arrow(C_W, W, label=r'$\tilde{%s}$' %sym.W)
_arrow(C_B, B, label=r'$\tilde{%s}$' %sym.B, color='g')
_scatter(C_W, color='b', label=r'$\tilde{%s}$' %sym.C_W)
_scatter(C_B, color='g', label=r'$\tilde{%s}$' %sym.C_B)
_scatter(w_i.take(arange(0, w_i.shape[0], n.K), axis=0), color = 'g')
legend(scatterpoints=1)
grid()
savefig('triangle_tilde.pdf')
\end{pylabcode}
As an anlytical approach seems feasible for simple shapes only, we propose a more general framework.
Assume that the body can be represented as a collection $\{\s{_idx(p)}\}$ of particles
\begin{equation*}
	\s{p} = \s{(V_p, rho_p, r_p)}	
\end{equation*}
such that
\begin{equation*}
	\begin{split}
\sum\limits_i^N \s{_idx(V_p)} = \s{V} \\ 
\sum\limits_i^N \s{_idx(V_p) * _idx(rho_p)} = \s{M}
\end{split}
\end{equation*}
Then the $\s{(Vi, rhoi)}$ may be replaced by $\s{(V_p, rho_p)}$ and equations (\ref{eq:maw}, \ref{eq:poa}) become summations.
The result is shown in Figure \ref{fg:fbdb}.  
\begin{figure}
\centering
\begin{subfigure}[t]{0.4\textwidth}
\includegraphics[width=\textwidth]{triangle.pdf}
\caption{Analytic approach}
\label{fg:fbda}
\end{subfigure}
~
\begin{subfigure}[t]{0.4\textwidth}
\includegraphics[width=\textwidth]{triangle_tilde.pdf}
\caption{Sampling approach with $N = \s{n.N}$ particles (every $\s{10}$-th displaced water particle drawn)}
\label{fg:fbdb}
\end{subfigure}
\caption{Free body diagrams illustrating buoyancy and gravity for a planar triangle of uniform normalized density $\s{Eq(rho/rho_W, n.rho)}$, assuming that the global water level is at $\s{Eq(z)}$ and not affected by the body.}
\label{fg:fbd}
\end{figure}
 \end{document}

