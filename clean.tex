\documentclass[12pt,a4paper]{article}
\usepackage{amsmath}  % required for {\operatorname}
\usepackage{amsfonts} % required for {\mathbb}
\usepackage{hyperref} % allows \href{url}{Label}, not yet made available on arch
\usepackage[english]{babel}
\usepackage{siunitx}
\usepackage[]{biblatex}
\usepackage[babel]{csquotes}
\usepackage[]{pythontex}
\newcommand{\s}{\sympy}
\newcommand{\p}[1]{\pyc{_p(#1)}}
\usepackage{todonotes}  
\usepackage{float}
\usepackage{graphicx}
\usepackage{caption}
\usepackage{subcaption}
\usepackage{dsfont}
\bibliography{ CognitiveScience2011ModelLearning.bib, BaraffRigidBodySimulation.bib }
\begin{document}
\tableofcontents

% set up sympy
\begin{sympycode}
sys.path.append(r'C:\Users\Murad\evrythg')
from sympy.physics.units import *
# clean_sym.py
from clean_sym import *
# _sympy.py 
from _sympy import _idx
# clean_numeric_values.py
import clean_numeric_values as n
\end{sympycode}

\pyc{import os}
\pyc{os.chdir(r'C:\Users\Murad\Dropbox\evrythg')}
\pyc{execfile('clean.py')}


\section{Task description}
Model the dynamics of a rigid body in an environment partly filled with water.
\section{Model description}
\subsection{Statics}
Assume that a rigid body is specified by its geometry $\pyc{_o(G)}$ and density $\s{rho}$, both assumed not to change over time.    Positions $\s{r}$ are described using a right handed world coordinate system aligned such, that the $\s{z}$-Axis is collinear to line of action of gravity.  When convenient we make use of a body coordinate system coordinates $\s{b}$.  Unit vectors along the axes are given as
\begin{equation*}
\s{[i, j, k]} = \s{eye(3)}
\end{equation*}

When infinitesimal volumes $\s{Vi}$ in the body are asssumed to have constant densities $\s{rhoi}$, infinesimal masses $\s{mi}$ are subject to weight forces $\s{Wi}$ as
\begin{equation*}
\s{Eq(mi, rhoi * Vi)} \;;\; \s{Wi} =  \s{g * k} \s{mi} \;;\; \s{Eq(g, 9.81 * meter/second**2)}
\end{equation*}
\begin{equation}
\s{m} = \int\limits_{\s{G}} \s{mi} \;;\; \s{W} = \int\limits_{\s{G}} \s{Wi}
\label{eq:maw}
\end{equation}
If the body displaces a volume $\s{D}$ of water it experiences an additional force
$$\s{B} = \s{g * rho_W *  k}\int\limits_{\s{D}} \s{Di} \;;\; \s{Eq(rho_W, 1000 * kilo * gram/meter**3)}$$
The points of application $\s{(C_B, C_W)}$ for the respective forces $\s{(B, W)}$ are
\begin{equation}
	\s{C_W} = \int\limits_{\s{G}} \s{mi * ri / m} \;;\; \s{C_B} = \int\limits_{\s{B}} \s{ri} 
\label{eq:poa}
\end{equation}
The body would be floating in equilibrium if
$$\s{Eq(W, B)} \;;\; \s{Eq(C_W, C_B)}$$
and no other forces were are present.  
However when $\overrightarrow{\s{C_B} \s{C_W}}$ is not aligned with the $\s{z}$-axis, the body will experience a torque 
$$\s{tau_B} = (\s{C_B-C_W}) \times \s{B}$$
Also in general, we need to assume that the body is affected by a total force
\begin{equation*}
\s{Eq(F, X + W + B)}
\end{equation*}
where $\s{X}$ accounts for any unknown forces.
\subsection{Dynamics}
To describe system dynamics we introduce a state vector $\s{s}$.
In general
\begin{equation}
	\begin{split}
		\s{Eq(F, f)} \\
		\s{Eq(sddot, h)}\\
\end{split}
\label{eq:eom}
\end{equation}
Following \cite{140505160511} we propose 
\begin{equation*}
	\s{(e['140505171627'])}
\end{equation*}
such that $(C_W(t), \s{q}})$ specifies the relation between body and world coordinate system, $\s{P}$ is the linear and $\s{L}$ the angular momentum.  
Then the idealized equations of motion are given as 
\begin{equation*}
	\s{e['140507215327']}
\end{equation*}
\section{Implementation}
To solve equations (\ref{eq:maw}, \ref{eq:poa}) indicator functions $\mathds{1}_{(.)}(.)$ decide if any coordinate $\s{r}$ is occupied by a infinitesimal volume $\s{Vi}$ at time $\s{t}$ and if so which density $\s{rhoi}$ is to be assigned to $\s{Vi}$.
As such functions are hard to define for the general case we start with simplified setups.  
If the rigid body is described by a planar triangle and $\s{X}$ is neglected we obtain the forces illustrated in Figure \ref{fg:fbda}
As an anlytical approach seems feasible for simple shapes only, we propose a more general framework.
Assume that the body can be represented as a collection $\{\s{_idx(p)}\}$ of particles
\begin{equation*}
	\s{p} = \s{(V_p, rho_p, r_p)}	
\end{equation*}
such that
\begin{equation*}
	\begin{split}
\sum\limits_i^N \s{_idx(V_p)} = \s{V} \\ 
\sum\limits_i^N \s{_idx(V_p) * _idx(rho_p)} = \s{m}
\end{split}
\end{equation*}
Then the $\s{(Vi, rhoi)}$ may be replaced by $\s{(V_p, rho_p)}$ and equations (\ref{eq:maw}, \ref{eq:poa}) become summations.
The result is shown in Figure \ref{fg:fbdb}.  
\begin{figure}
\centering
\begin{subfigure}[t]{0.4\textwidth}
\includegraphics[width=\textwidth]{triangle.pdf}
\caption{Analytic approach}
\label{fg:fbda}
\end{subfigure}
~
\begin{subfigure}[t]{0.4\textwidth}
\includegraphics[width=\textwidth]{triangle_tilde.pdf}
\caption{Sampling approach with $N = \s{n.N}$ particles (every $\s{10}$th displaced water particle drawn)}
\label{fg:fbdb}
\end{subfigure}
\caption{Free body diagrams illustrating buoyancy and gravity for a planar triangle of uniform normalized density $\s{Eq(rho/rho_W, n.rho)}$, assuming that the global water level is at $\s{Eq(z)}$ and not affected by the body.}
\label{fg:fbd}
\end{figure}
\section{Control}
\todo[inline]{Da waer zunaechst mal ein Regelungsziel zu definieren.}
\blockquote{
1. Wir fixieren das Boot, so dass Bewegungen nur in einer oder zwei Dimensionen m"oglich sind. \\
\todo[inline]{Hier waer Drehung um die Laengsachse/Querachse am interessantesten oder?}
\todo[inline]{Dann stellt sich die Frage ob man die Translation entlang der $\s{z}$-Achse nicht zunaechst auch vernachlaessigen kann, da wir sie durch unsere Kontrollmasse ohnehin in weitaus geringerem Umfang beeinflussen koennen als die Rotationen.}
2. Wir messen die f"ur diese Freiheitsgrade relevanten IMU-Werte aus. \\
\todo[inline]{Benutzen wir dazu die Kontrollmasse? Kalibrierung?}
3. Wir machen eine (bspw. SVM-)Regression mit den IMU-Werten und erhalten mit der Regressionsgeraden eine Pr"adiktion (oder ein Modell), wie sich das Boot bewegen wird. \\
\todo[inline]{Wie wuerde die Input-Output Relation formal aussehen?}
4. Mithilfe dieser Pr"adiktion, die wir zu jedem Zeitpunkt t erhalten, regeln wir die Position der Masse. \\
\todo[inline]{Entspricht einem inverse model in \cite{140505160108}, oder?  Sind wir sicher dass das Problem dann nicht ill-posed ist?}
5. Wir spielen die Simulation mit deiner gew"ahlten Funktion X durch. \\
6. Aus der Simulation erhalten wir eine Pr"adiktion (oder ein Modell) f"ur jeden Zeitpunkt t. \\ 
7. Mithilfe dieser Pr"adiktion, die wir zu jedem Zeitpunkt t erhalten, regeln wir die Position der Masse. \\ 
8. Wir vergleichen die beiden Vorangehensweisen.
}
\todo[inline]{Continue reading \cite{140505160108}}
\printbibliography
 \end{document}

