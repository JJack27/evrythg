\documentclass[12pt,a4paper]{article}
\usepackage{amsmath}  % required for {operatorname}
\usepackage{amsfonts} % required for {mathbb}
% \usepackage{trfsigns} % required for laplace transformation signs, not available in EICOM
% \usepackage{hyperref} % alows \href{url}{Label}, not yet made available on arch
\usepackage{siunitx} 
\begin{document}

\tableofcontents

\begin{sympycode}

from sympy.physics.units import *
# Cartesian unit vectors
i = Symbol(r'\mathbf{i}') # unit vector along x-Axis
j = Symbol(r'\mathbf{j}') # unit vector along y-Axis
k = Symbol(r'\mathbf{k}', commutative=False) # unit vector along z-Axis
V = Symbol(r'\mathcal{V}') # Volume
Vi = Symbol(r'\operatorname{d}%s' %(V)) # infinitesimal  volume
D = Symbol(r'\mathcal{D}') # Volume of displaced water
Di = Symbol(r'\operatorname{d}%s' %(D)) # infinitesimal  volume
rho = Symbol(r'\rho') # mass density
rho_W = Symbol(r'\rho_W') # mass density of water
rhoi = Symbol(r'\operatorname{d}%s' %(rho)) # infinitesimal mass density
M = Symbol(r'M') # mass
W = Symbol(r'W') # gravity force
B = Symbol(r'B') # buoyant force
X = Symbol(r'X') # unknown force
F = Symbol(r'F') # total force
Wi = Symbol(r'\operatorname{d} %s' %(W)) # infinitesimal forces
Mi = Symbol(r'\operatorname{d} %s' %(m)) # infinitesimal masses
g = Symbol(r'g') # earth acceleration
s = Symbol(r's') # state vector
sdot = Derivative(s, Symbol('t'))
sddot = Derivative(s, Symbol('t'),2)
f = Function('\operatorname{f}')(s, sdot) # total force as a function
h = Function('\operatorname{h}')(F) # change of state 
ss = Symbol(r's^*') # state vector in equilibrium
P_xz = Symbol(r'\mathcal{P}_{x,z}') # x-z-Plane
A_xz = Symbol(r'A_{x,z}') # Area in the x-z-plane 
A_i = Symbol(r'A_{i}') # Area of a planar shape
S_i = Symbol(r'\mathcal{S}_{i}') # Planar shape 
T_i = Symbol(r'\mathcal{T}_{i}') # Planar triangle
VA = Symbol(r'A') # Vertex 
VB = Symbol(r'B') # Vertex 
VC = Symbol(r'C') # Vertex 
c = Symbol(r'\mathsf{c}', commutative=False) # some constant
exec(open('C:\Users\Murad\evrythg\pythontex_functions.py').read())
\end{sympycode}
\section{Task description}
Model the dynamics of a rigid body in an environment partly filled with water.
\section{Model description}
The rigid body is specified by its' volume $\s{V}$ and density $\s{rho}$, which are assumed not to change over time.  To describe system dynamics we introduce a state vector $\s{s}$.  Positions and directions are described using a right handed coordinate system aligned such that the $z$-Axis is collinear to line of action of gravity.  Unit vectors along the axes are given as
$$\s{[i, j, k]} = \s{eye(3)}$$
When infinitesimal volumes $\s{Vi}$ are asssumed to have constant densities $\s{rhoi}$ we have 
$$\s{Eq(Mi, rhoi * Vi)} \;;\; \s{Wi} =  \s{g * k} \s{Mi} \;;\; \s{Eq(g, 9.81 * meter/second**2)}$$
$$\s{M} = \int\limits_{\s{V}} \s{Mi} \;;\; \s{W} = \int\limits_{\s{V}} \s{Wi}$$ 
If the body displaces a volume $\s{D}$ of water it experiences an additional force
$$\s{B} = \s{g * rho_W *  k}\int\limits_{\s{D}} \s{Di} \;;\; \s{Eq(rho_W, 1000 * kilo * gram/meter**3)}$$
The body would be floating in equilibrium if
$$\s{Eq(W, B)}$$ and no other forces were are present.  
Otherwise it is affected by a total force
$$\s{Eq(F, X + W + B)}$$
where $\s{X}$ accounts for any unknown forces.
In general
\begin{equation}
	\begin{split}
		\s{Eq(F, f)} \\
		\s{Eq(sddot, h)} \\
\end{split}
	\label{eq:eom}
\end{equation}

\section{Implementation}
To obtain implementations for $\s{f.func}, \s{h.func}$ we require a suitable representation for $\s{tuple(V, s, rho)}$.
Initially we impose stricts constraints on $(\s{V}, \s{rho})$ which may be relaxed later on.
Let 
\begin{itemize}
	\item $\s{V}$ be symmetric with respect to $\s{P_xz} = \operatorname{span}(\s{i}, \s{k})$
	\item $\s{rho}$ be constant over $\s{V}$
\end{itemize}
Then 
$$\s{V} = \s{c} \s{A_xz} \;;\; \s{A_xz} = \int\limits_{\s{P_xz}} \s{Vi}$$
Furtheron assume that 
$$\s{A_xz} = \sum\limits_i\s{A_i}$$
where $\s{A_i}$ is the area of a planar shape $S_i \in \s{P_xz}$ and $\s{c}$ is a constant.  
One particular class of shapes $\s{S_i}$ is that of triangles $\s{T_i}$ specified by vertices $\s{tuple(VA, VB, VC)}$
\input{post.tex}

