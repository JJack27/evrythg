\documentclass[12pt,a4paper]{article}
\usepackage{amsmath}  % required for {operatorname}
\usepackage{amsfonts} % required for {mathbb}
% \usepackage{trfsigns} % required for laplace transformation signs, not available in EICOM
% \usepackage{hyperref} % alows \href{url}{Label}, not yet made available on arch
\usepackage{siunitx} 
\usepackage{pythontex}
\begin{document}

\tableofcontents

% set up sympy
\begin{sympycode}
sys.path.append(r'C:\Users\Murad\evrythg')
from clean_sym import *
from pythontex_functions import *
\end{sympycode}

% set up pylab
\begin{pylabcode}
sys.path.append(r'C:\Users\Murad\evrythg')
from numpy import array
from wrpr import _line
from wrpr import _scatter
from wrpr import _arrow
from wrpr import _pp
from cg import A_T
from cg import  auw
import clean_sym as sym
from sympy.geometry import Point
from sympy.geometry import Triangle
\end{pylabcode}


\section{Task description}
Model the dynamics of a rigid body in an environment partly filled with water.
\section{Model description}
The rigid body is specified by its' volume $\s{V}$ and density $\s{rho}$, which are assumed not to change over time.  To describe system dynamics we introduce a state vector $\s{s}$.  Positions and directions are described using a right handed coordinate system aligned such that the $z$-Axis is collinear to line of action of gravity.  Unit vectors along the axes are given as
$$\s{[i, j, k]} = \s{eye(3)}$$
When infinitesimal volumes $\s{Vi}$ are asssumed to have constant densities $\s{rhoi}$ we have 
$$\s{Eq(Mi, rhoi * Vi)} \;;\; \s{Wi} =  \s{g * k} \s{Mi} \;;\; \s{Eq(g, 9.81 * meter/second**2)}$$
$$\s{M} = \int\limits_{\s{V}} \s{Mi} \;;\; \s{W} = \int\limits_{\s{V}} \s{Wi}$$ 
If the body displaces a volume $\s{D}$ of water it experiences an additional force
$$\s{B} = \s{g * rho_W *  k}\int\limits_{\s{D}} \s{Di} \;;\; \s{Eq(rho_W, 1000 * kilo * gram/meter**3)}$$
The body would be floating in equilibrium if
$$\s{Eq(W, B)}$$ and no other forces were are present.  
Otherwise it is affected by a total force
$$\s{Eq(F, X + W + B)}$$
where $\s{X}$ accounts for any unknown forces.
In general
\begin{equation}
	\begin{split}
		\s{Eq(F, f)} \\
		\s{Eq(sddot, h)} \\
\end{split}
	\label{eq:eom}
\end{equation}

\section{Implementation}
To obtain implementations for $\s{f.func}, \s{h.func}$ we require a suitable representation for $\s{tuple(V, s, rho)}$.
Initially we impose stricts constraints on $(\s{V}, \s{rho})$ which may be relaxed later on.
Let 
\begin{itemize}
	\item $\s{V}$ be symmetric with respect to $\s{P_xz} = \operatorname{span}(\s{i}, \s{k})$
	\item $\s{rho}$ be constant over $\s{V}$
\end{itemize}
Then 
$$\s{V} = \s{c} \s{A_xz} \;;\; \s{A_xz} = \int\limits_{\s{P_xz}} \s{Vi}$$
Furtheron assume that 
$$\s{A_xz} = \sum\limits_i\s{A_i}$$
where $\s{A_i}$ is the area of a planar shape $S_i \in \s{P_xz}$ and $\s{c}$ is a constant.  
One particular class of shapes $\s{S_i}$ is that of triangles $\s{T_i}$ specified by vertices $\s{tuple(VA, VB, VC)}$

% \begin{pylabcode}
% T = array([[4, 0], [3, 4], [0, 1]]) + array([0, -2])
% C = mean(T, axis=0)
% figure()
% _line(T)
% savefig('triangle.pdf')
% \end{pylabcode}

\begin{pylabcode}
o = Point(-1, -1)
p = Point(1, -1)
q = Point(0, 1)
T = Triangle(o, p, q).rotate(pi/5.)
C_W = array([T.centroid.x, T.centroid.y]).astype(float)
D = auw(T)
C_B = array([D.centroid.x, D.centroid.y]).astype(float)
rho = 0.7
g = array([0, -1])
W =  float(T.area) * rho
B =  - float(D.area) * 1 
figure()
_pp(T)
_pp(D, color='g')
_arrow(C_W, g*W, label='$%s$' %sym.W)
_arrow(C_B, g*B, label='$%s$' %sym.B, color='g')
_scatter(C_W, color='b', label='$%s$' %sym.C_W)
_scatter(C_B, color='g', label='$%s$' %sym.C_B)
legend(scatterpoints=1)
grid()
savefig('triangle.pdf')
\end{pylabcode}

\begin{figure}[]
	\centering
	\includegraphics{pythontex-files-clean/triangle.pdf}
	\caption{Free body diagram}
	\label{fig:fbd}
\end{figure}
 \end{document}


