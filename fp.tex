\documentclass[12pt,a4paper]{article}
\usepackage{amsmath}  % required for {operatorname}
\usepackage{amsfonts} % required for {mathbb}
% \usepackage{trfsigns} % required for laplace transformation signs, not available in EICOM
% \usepackage{hyperref} % alows \href{url}{Label}, not yet made available on arch
\usepackage{siunitx} 
\usepackage{pythontex}
\begin{document}

\tableofcontents
\section{Task description}
Model the dynamic behavior of a rigid body in an environment partly filled with water.

\section{Model description}
We assume that the rigid body's geometry is defined by an abstract constraint set $\matcal{G}$ and has a known Mass $M$.  
Here $z$ is the world space location of the body's centroid given as
$$z = \frac{1}{M} \int\limits_{\mathcal{G}} \rho(\zeta) \zeta \operatorname{d} \zeta, $$
(cf.  \href{http://en.wikipedia.org/w/index.php?title=Center\_of\_mass&action=edit&section=4}{url}),
$\mathcal{G}$ describes its' geometry and $M$ is its' total mass. 
Gravity and the presence of water induce a total force $F(z)$, leading to the differential equation
\begin{equation}
\ddot{z} = \frac{F(z)}{M} + R(\dot{z}),  
\label{eq:eom}
\end{equation}
where the second term $R(z)$ accounts for friction.
We assume that the body is able to float in equilibrium postion i.e.
$$\exists z^* \; F(z^*) = 0$$
\todo[inline]{How to represent the body:\\  - continue with my own triangle code and the particle approach? \\ - explore the Python bindings for the CGAL algorithm? (cf.  \href{https://code.google.com/p/cgal-bindings/}{url})}

\section{Particle Representation}
Arbitrary geometries $\mathcal{G}$ may be represented as a collection $\mathcal{T}$ of triangles $T_j,$
$$T = \left\{z_A, z_B, z_C\right\}$$
defined by the vertex coordinates $z_A, z_B, z_C$
(cf.  \href{Burkardt-ComputationalGeometryLab-01-Triangles.pdf}{p. 2})
To solve the equation of motion \ref{eq:eom} we assume that the dynamics of $B$ can be described by studying collections $\mathcal{P}$, $|\mathcal{P}| = N$ of particles $p_i \in \mathcal{G}$, $p_i = \left(\rho_i, V_i, z_i\right)$ with uniform mass densities $\rho_i$, Volumes $V_i$ and positions $z_i \in \mathcal{Z}$ subject to the constraint:
$$\frac{\partial z_i - z_j}{\partial t} = 0 \; \forall p_i, p_j \in \mathcal{P}$$
and affected by forces $F_i(z_i)$
Then
$$F(z) = \sum\limits_i^N F_i(z_i)$$
The volumes of the individual particles are determined to fullfil
$$\sum\limits_i^N \rho_i V_i = M$$
In an randomize approach the $z_i$ are determined through sampling over $A_\mathcal{G}$.  For physical consistency we require
$$\lim\limits_{N \to \infty} \sum\limits_i^N \rho_i V_i z_i = z$$
For finite sample sizes $N < \infty$ we can validate the sampling method by defining an indicator function 
\begin{equation}
\mathsf{I}: \left\{(T, z_i)\right\} \to \left\{0, 1\right\}
\label{eq:in}
\end{equation}
and checking that
$$\frac{\sum\limits_i J(T, p_i)}{A_T} \approx \frac{N}{A_{\mathcal{B}}}$$
for any $T$ with area $A_T$.  
Having defined the $p_i$ we assume that earth acceleration is given as
$$a_g = - \num{9.81}  \si{\metre\per\square\second}$$ 
and define an indicator function 
$$\mathsf{J}: \mathcal{Z} \to \left\{ 0, 1\right\}$$
to decide if a particle is under water surface.  
Using Archimedes' law
(cf.  \href{LautrupPhysicsOfContinuousMatter}{p. 68})
we have 
$$f_i(z_i) = a_g(- J(z_i) V_i \rho_w + V_i \rho_i)$$
where $\rho_w = \num{1000} \si{\kilo\gram\per\cubic\metre}$ is the mass density of water.

\section{Implementation}
To continue it is is useful to introduce barycentric coordinates:
$$\zeta_A = \frac{d(z_i, \{z_B, z_C\})} {z_A, \{z_B, z_C\}}, \; 
\zeta_B = \frac{d(z_i, \{z_C, z_A\})} {z_B, \{z_C, z_A\}}, \;
\zeta_C = \frac{d(z_i, \{z_A, z_B\})} {z_C, \{z_A, z_B\}}$$
(cf.  \href{Burkardt-ComputationalGeometryLab-02-BarycentricCoordinatesinTriangles.pdf}{p. 6})
where $d(z_i, L)$ is defined as the signed distance from $z_i$ to a Line $L = \{z_P, z_Q\}$ connecting Points $P, Q$.
(cf.  \href{Burkardt-ComputationalGeometryLab-02-BarycentricCoordinatesinTriangles.pdf}{p. 2})
Then a possible implementation for the indicator function $\mathsf{I}$ (c.f. \ref{eq:in}) is
$$\mathsf{I}: \left\{(T, z_i)\right\} \mapsto (\zeta_A(T, z_i) \geq 0) \land (\zeta_B(T, z_i) \geq 0) \land (\zeta_C(T, z_i) \geq 0)$$
(cf.  \href{Burkardt-ComputationalGeometryLab-02-BarycentricCoordinatesinTriangles.pdf}{p. 6})
Following the approach suggested by 
(cf.  \href{Burkardt-ComputationalGeometryLab-03-MonteCarloOnTriangles.pdf}{p. 4})
we sample 
$$r, s \sim \mathcal{U}^{\times 2}(0; 1)$$
and obtain
$$\zeta^T = [1 - \sqrt{s}, (1 - r) \sqrt{s}, r \sqrt{s}], $$
$$z_i = [z_A, z_B, z_c] \zeta$$
 \end{document}


