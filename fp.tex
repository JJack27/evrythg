\documentclass[12pt,a4paper]{article}
\usepackage{amsmath}  % required for {operatorname}
\usepackage{amsfonts} % required for {mathbb}
% \usepackage{trfsigns} % required for laplace transformation signs, not available in EICOM
% \usepackage{hyperref} % alows \href{url}{Label}, not yet made available on arch
\usepackage{siunitx} 
\usepackage{pythontex}
\begin{document}

\section{Task description}
Model the dynamic behavior of a rigid body in an environment partly filled with water.

\section{Model description}
Represent the body as $B = \left(m, \mathcal{G}, z\right)$ with mass $m$, geometry $\mathcal{G}$ and position $z$. 
Gravity and the presence of water induce a total force $F(z)$.  

\section{Implementation}
Assume that the dynamics of $B$  can be described by studying a collection $\mathcal{C}$, $|\mathcal{C}| = N$ of particles $p_i \in \mathcal{C}$, $p_i = \left(\rho_i, V_i, z_i\right)$ with uniform mass densities $\rho_i$, Volumes $V_i$ and positions $z_i \in \mathcal{Z}$ subject to the constraints:
$$\sum\limits_i^N \rho_i V_i = m$$
$$\frac{\partial z_i - z_j}{\partial t} = 0 \; \forall p_i, p_j \in \mathcal{C}$$
$$\sum\limits_i^N F_i(z_i) = F(z)$$
We assume that earth acceleration is given as
$$a_g = - \num{9.81}  \si{\metre\per\square\second}$$ 
and define an indicator function
$$\operatorname{I}: \mathcal{Z} \to \left\{ 0, 1\right\}$$
Using Archimedes' law we have 
$$f_i(z_i) = a_g(- I(z_i) V_i \rho_w + V_i \rho_i)$$
where $\rho_w = \num{1000} \si{\kilo\gram\per\cubic\metre}$ is the mass density of water.
 \end{document}

