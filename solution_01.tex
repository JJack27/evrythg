\documentclass[12pt,a4paper]{article}
\usepackage{amsmath}  % required for {operatorname}
\usepackage{amsfonts} % required for {mathbb}
% \usepackage{trfsigns} % required for laplace transformation signs, not available in EICOM
% \usepackage{hyperref} % alows \href{url}{Label}, not yet made available on arch
\usepackage{siunitx} 
\begin{document}

\pyc{execfile('solution_01.py')}

\begin{verbatim}
Michael Baumann
#03636497
\end{verbatim}

\begin{figure}[ht]
\includegraphics[width=\textwidth]{140423111519.pdf}
\caption{\pyc{_input(140423111519)}}
\end{figure}

\pyc{_input(140423105502)}
We have
$$\alpha = 0.5 \;;\; r = 1 \;;\; \forall K$$
Using equation (2.7) from the online textbook
$$Q_k = (1 - \alpha)^k Q_0 + \sum\limits_{i=1}^k \alpha (1-\alpha)^{k-i} r_i$$
becomes
$$Q_k = 0.5  \sum\limits_{i=1}^k 0.5^{k-i}$$
Using the geometric series we obtain
$$Q_k = 1 - 0.5^k$$
\pyc{t_140423105502()}

\begin{figure}[ht]
\includegraphics[width=\textwidth]{140424090205.pdf}
\caption{\pyc{_input(140424090205)}}
\end{figure}

\begin{figure}[ht]
\includegraphics[width=\textwidth]{140424090951.pdf}
\caption{\pyc{_input(140424090951)}}
\end{figure}

Measure the error using the squared distance.
$$\epsilon = \sum\limits_{k=12}^{18} Q_k - r_k$$
\pyc{t_140424090951()}

\begin{figure}[ht]
\includegraphics[width=\textwidth]{140424105951.pdf}
\caption{\pyc{_input(140424105951)}
\pyc{_input(140424133343)}
Initially a step size adapted in this way leads to fast learning.
Contrary to a large fixed step size it does not oscillate when the target is changing.
\pyc{_input(140424133907)}
The learning/adaption rate is continously decreasing over time and 0 in the limit.
Thus using  this stepsize an agent would not be able to follow a changing target after some time.
}
\end{figure}

\begin{figure}
	\begin{subfigure}[ht]{0.49\textwidth}
	\includegraphics[width=\textwidth]{140424135248-0_5.pdf}
	\end{subfigure}
	~
	\begin{subfigure}[ht]{0.49\textwidth}
	\includegraphics[width=\textwidth]{1404241352481_5.pdf}
	\end{subfigure} 

	\begin{subfigure}[ht]{0.49\textwidth}
	\includegraphics[width=\textwidth]{1404241352482_0.pdf}
	\end{subfigure}
	~
	\begin{subfigure}[ht]{0.49\textwidth}
	\includegraphics[width=\textwidth]{1404241352482_5.pdf}
	\end{subfigure} 
\caption{Exploring steps sizes $\alpha$}
\label{fig:140424135248}
\end{figure}
\pyc{_input(140424135248)}
Based on the plots shown in figure \ref{fig:140424135248} the safe range seems to be 
$$\alpha \in \left[0, 2 \right]$$
(where $\alpha = 0$ seems rather pointless)
\input{post.tex}

